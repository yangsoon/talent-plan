\documentclass[UTF8]{ctexart}
%%%%%%%%%%%%%%%%%%%%%%%%%%%== 引入宏 ==%%%%%%%%%%%%%%%%%%%%%%%%%%%%%
\usepackage{cite}
\usepackage{amsmath}	% 使用数学公式
\usepackage{graphicx}	% 插入图片/PDF/EPS 等图像
\usepackage{subfigure}	% 使用子图像或者子表格
\usepackage{geometry}	% 设置页边距
\usepackage{fancyhdr}	% 设置页眉页脚
\usepackage{setspace}	% 设置行间距
\usepackage{hyperref}	% 让生成的文章目录有链接,点击时会自动跳转到该章节
\usepackage{url}
\usepackage{caption2}
\usepackage{forest}
\usepackage{float}

\def\ojoin{\setbox0=\hbox{$\bowtie$}%
  \rule[-.02ex]{.25em}{.4pt}\llap{\rule[\ht0]{.25em}{.4pt}}}
\def\leftouterjoin{\mathbin{\ojoin\mkern-5.8mu\bowtie}}
\def\rightouterjoin{\mathbin{\bowtie\mkern-5.8mu\ojoin}}
\def\fullouterjoin{\mathbin{\ojoin\mkern-5.8mu\bowtie\mkern-5.8mu\ojoin}}

%%%%%%%%%%%%%%%%%%%%%%%%%%== 设置全局环境 ==%%%%%%%%%%%%%%%%%%%%%%%%%%%%
% [geometry] 设置页边距
\geometry{top=2.6cm, bottom=2.6cm, left=2.45cm, right=2.45cm, headsep=0.4cm, foot=1.12cm}
% 设置行间距为 1.5 倍行距
\onehalfspacing
% 设置页眉页脚
\pagestyle{fancy}
%\lhead{左头标}
%\chead{\today}
%\rhead{152xxxxxxxx}
\lfoot{}
\cfoot{\thepage}
\rfoot{}
%\renewcommand{\headrulewidth}{0.4pt}
%\renewcommand{\headwidth}{\textwidth}
%\renewcommand{\footrulewidth}{0pt}

%%%%%%%%%%%%%%%%%%%%%%%%%%== 自定义命令  ==%%%%%%%%%%%%%%%%%%%%%%%%%%%%%%
% 此行使文献引用以上标形式显示
\newcommand{\supercite}[1]{\textsuperscript{\cite{#1}}}
% 此行使section中的图、表、公式编号以A-B的形式显示
\renewcommand{\thetable}{\arabic{section}-\arabic{table}}
\renewcommand{\thefigure}{\arabic{section}-\arabic{figure}}
\renewcommand{\theequation}{\arabic{section}-\arabic{equation}}
% 此行使图注、表注与编号之间的分隔符缺省,默认是冒号:
\renewcommand{\captionlabeldelim}{~}

%===================================== 标题设置  ==========================================
% \heiti \kaishu 为字体设置,ctex 会自动根据操作系统加载字体
\title{\huge{\heiti Talent-Plan Section 3}}
\author{\small{\kaishu 宋阳}\\[2pt]
\small{\kaishu 北京航空航天大学}\\[2pt]
\small{Email:}
\url{yangsoonlx@gmail.com}
}
\date{} % 去除默认日期
%\date{\today}

%===================================== 正文区域  ==========================================
\begin{document}
\maketitle
% \tableofcontents % 目录内容,注释取消掉可实现目录

\begin{flushleft}
\textbf{课程目标}:熟悉数据库基础知识 \\[8pt]
\end{flushleft}
\section{课程作业}\label{sec1}

select $t_1.a$, count(*), avg($t_1.b$) from $t_1$ left outer join $t_2$ on $t_1.a=t_2.a$ group by $t_1.a$,请给出
所有可能的逻辑执行计划(画出Plan树),并分析$t_1$的数据分布对各种逻辑执行计划执行性能的影响。
查询的初步逻辑查询计划如下图所示: \\
\begin{figure}[H] 
  \begin{center}
    \fontsize{15pt}{15pt}\selectfont
    \begin{forest}
        [, phantom, s sep = 1cm
            [$\gamma_{t_1.a,\,count(*),\,avg(t_1.b)}$
              [$\mathop{\leftouterjoin}\limits_{t_1.a\,=\,t_2.a}$  
                  [$t_1$
                  ]
                  [$t_2$]
              ]
            ]
        ]
    \end{forest}
  \end{center}
  \caption{初步逻辑查询图} \label{tree1}
\end{figure}
\section{逻辑执行计划分析}\label{sec2}

逻辑优化主要是基于规则的优化,数据库逻辑优化规则包括:列裁剪,最大最小消除,投影消除,谓词下推等等。
本次课程作业的sql语句包括连接和聚合操作,针对这类sql语句可以使用聚合消除,外连接消除等逻辑优化规则。
逻辑优化会针对sql语句中算子的不同性质进行不同的优化操作。所以接下来开始讨论在下面这几种不同的组合下可以进行的
逻辑优化操作。

\begin{enumerate}
	\item 属性$t_{1}.a$在表$t_{1}$中不具有唯一性,属性$t_{2}.a$在表$t_{2}$中也不具有唯一性;
	\item 属性$t_{1}.a$在表$t_{1}$中具有唯一性,属性$t_{2}.a$在表$t_{2}$中不具有唯一性;
	\item 属性$t_{1}.a$在表$t_{1}$中不具有唯一性,属性$t_{2}.a$在表$t_{2}$中具有唯一性;
	\item 属性$t_{1}.a$在表$t_{1}$中具有唯一性,属性$t_{2}.a$在表$t_{2}$中具有唯一性;
\end{enumerate}

\subsection{属性$t_{1}.a$和属性$t_{2}.a$均不具有唯一性}
第一种组合是最普通的组合,即属性$t_{1}.a$在表$t_{1}$中不具有唯一性,属性$t_{2}.a$在表$t_{2}$中也不具有唯一性。这时候,只能对图\ref{tree1}中的最初始的执行计划进行优化:
从sql语句中可以看到,最终的输出结果只涉及到了表$t_1$中的属性a和属性b,表$t_2$只是用来做连接操作的。所以我们使用列裁剪优化规则,裁剪掉用不上的列。优化后的逻辑查询计划如图\ref{tree2}所示:

\begin{figure}[H] 
  \begin{center}
    \fontsize{15pt}{15pt}\selectfont
    \begin{forest}
        [, phantom, s sep = 1cm
            [$\gamma_{t_1.a,\,count(*),\,avg(t_1.b)}$
              [$\mathop{\leftouterjoin}\limits_{t_1.a\,=\,t_2.a}$  
                  [$\pi_{t_1.a, \, t_1.b}$
                    [$t_1$]
                  ]
                  [$\pi_{t_2.a}$
                    [$t_2$]
                  ]
              ]
            ]
        ]
    \end{forest}
  \end{center}
  \caption{属性$t_{1}.a$和属性$t_{2}.a$均不具有唯一性时的逻辑查询图} \label{tree2}
\end{figure}

通过列裁剪可以减少不必要的IO,优化语句执行。

\subsection{属性$t_{1}.a$具有唯一性,属性$t_{2}.a$不具有唯一性}
在这种情况下,属性$t_1.a$具有唯一性,所以$(t_1.a, \, t_1.b)$的组合也是唯一的,所以$avg(t1.b)$可以直接退化为$t_1.b$。
同样的为了减少不必要的IO,会对表$t_1$和表$t_2$进行列裁剪。优化后的逻辑查询计划图如图\ref{tree3}所示:
\begin{figure}[H] 
  \begin{center}
    \fontsize{15pt}{15pt}\selectfont
    \begin{forest}
        [, phantom, s sep = 1cm
            [$\gamma_{t_1.a,\,count(*),\,t_1.b}$
              [$\mathop{\leftouterjoin}\limits_{t_1.a\,=\,t_2.a}$  
                  [$\pi_{t_1.a, \, t_1.b}$
                    [$t_1$]
                  ]
                  [$\pi_{t_2.a}$
                    [$t_2$]
                  ]
              ]
            ]
        ]
    \end{forest}
  \end{center}
  \caption{属性$t_{1}.a$和属性$t_{2}.a$均不具有唯一性时的逻辑查询图} \label{tree3}
\end{figure}

\subsection{属性$t_{1}.a$不具有唯一性,属性$t_{2}.a$具有唯一性}
当$t_{2}.a$具有唯一性属性时,sql语句中的左外连接算子可以进行消除。
对于表$t_{1}$的每一行来说,,经过左外连接之后,最终只产生一行连接结果,
而且,最终输出的结果只需要表$t_1$中的属性。因此,优化后的逻辑查询结果如图\ref{tree4}
所示:
\begin{figure}[H] 
  \begin{center}
    \fontsize{15pt}{15pt}\selectfont
    \begin{forest}
        [, phantom, s sep = 1cm
            [$\gamma_{t_1.a,\,count(*),\,avg(t_1.b)}$
              [$t_1$]
            ]
        ]
    \end{forest}
  \end{center}
  \caption{属性$t_{1}.a$和属性$t_{2}.a$均不具有唯一性时的逻辑查询图} \label{tree4}
\end{figure}

\subsection{属性$t_{1}.a$和属性$t_{2}.a$均具有唯一性}
首先根据上面的结果,我们可以做外连接消除优化。又因为$t_1$也具有唯一性,所以可以继续做聚集消除操作
因此,优化后的逻辑查询结果如图\ref{tree5}所示:
\begin{figure}[H] 
  \begin{center}
    \fontsize{15pt}{15pt}\selectfont
    \begin{forest}
        [, phantom, s sep = 1cm
            [$\pi_{t_1.a,\,1,\,t_1.b}$
                [$t_1$]
            ]
        ]
    \end{forest}
  \end{center}
  \caption{属性$t_{1}.a$和属性$t_{2}.a$均不具有唯一性时的逻辑查询图} \label{tree5}
\end{figure}


\section{物理执行计划分析}\label{sec3}
\subsection{文件数据载入}
实验的所有数据都存在文件中,我们共有10000个轨迹数据文件,每个轨迹数据文件中约有1000条轨迹数据,每条轨迹数据约有60-100个点。寻找轨迹的相似轨迹,同一时间我们只关心两条轨迹的数据,因此:

\begin{itemize}
	\item 在加载目标轨迹时,是先从索引文件中读取到轨迹标号的行列,然后读取行对于的数据文件名,加载目标轨迹数据到内存。
	\item 在加载当前对比轨迹时,则直接从数据文件中读取轨迹数据,与目标轨迹进行对比计算。
	\item 并行计算时,在每个进程里都进行以上两个操作,不采用由主进程分发形式。
\end{itemize}

\end{document}